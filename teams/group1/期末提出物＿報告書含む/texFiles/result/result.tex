\chapter{結果}
\section{プロジェクトの成果}
本プロジェクトの前期の活動及び成果として、プロジェクト開始後,店員ロボットを制作するにあたっての問題点や役割についてディスカッションを行い「動き」「機能」「外見」の3つの観点に着目した.その中においてグループ1では「動き」に注目し,「動き」の動作を考察する上で,どんなに機能が優れていても,無機質な店員ロボットはいないのと同じであり,自分からお客さんに話しかけてもらうためのアプローチが必要であると考察した.また,今までのロボットらしくない動作を改善するため,実際の店員の動きを解析しました.そして,大きく自発動作と反応動作の2つ動作に分けることができると考察した.中間発表のアンケート結果より,「動き」の観点から課題解決を図る方向性に問題はなかった.質疑応答の時間に動画を流したことにより,肝心な質疑応答の時間が少なくなってしまい,質疑応答に回答できなかった人の疑問点を解消できず終わらせてしまった.他に,明確データなどがなく具体性に乏しさを感じさせるものとなってしまった.
\begin{flushright}
(※文責:木島拓海)
\end{flushright}
\section{プロジェクトにおける自分の役割}
本プロジェクトの前期の各個人の担当課題の成果は以下のようになった.
\begin{description}
   \item[伊藤壱]\mbox{}
   \item  電子回路を中心に学習,設計
      \begin{itemize}
     \item 直流回路と並列回路に纏わる電流、電圧の計算とコンダクタンスについて.
      \end{itemize}
      \begin{itemize}
     \item 容量とインダクタの特徴とそれを表す数式、回路記号について.
      \end{itemize}
      
   \item[木島拓海]\mbox{}
      \item リンク機構を中心とした,動きを実現する機構の学習,設計
      \begin{itemize}
      \item ロボット工作を購入し,組み立て機構や動きを学習
      \end{itemize}

   \item[藤内悠]\mbox{}
    \item  歯車設計などを中心とした,動きを実現する機構の学習,設計
      \begin{itemize}
      \item 腕の機構について
      \end{itemize}
      \begin{itemize}
      \item 遊星機構やユニバーサルジョイントを利用することの検討
      \end{itemize}
      \begin{itemize}
      \item be@brickの3dモデルを参考にしつつ内部の大まかな構想
      \end{itemize}

   \item[宮嶋佑]\mbox{}
     \item Fusion360による3D CADの学習,設計
      \begin{itemize}
      \item ロボットを3D CADで試作し,駆動域,モータが実装可能かの確認
      \item Fusion 360の学習:マスターガイドを参照
      \end{itemize}
\end{description}

\begin{flushright}
(※文責:木島拓海)
\end{flushright}

\section{今後の課題}
本プロジェクトのグループ1の目的である「先手を打つコミュニケーション」を実現するために,待ち動作を充実させることでもう一つのコンセプトである「先手を打つコミュニケーション」と対立することにならないように,待ち動作の工夫が必要である.さらに,ロボットの頭を撫でるという行為はコロナウイルス の流行している中では受け入れられづらい可能性があり,工夫や改善する必要がある.また,店員が果たすべき振る舞いとはどのようなものかを実際の店員の観察や認知心理に基づく理由を含め考察し,具体的にどのような動作があるのかといったことを実現可能な範囲で挙げること,その動作を全て実現可能とする店員ロボットの外観及び内面機構の設計を行う.今後の展望としては,3Dモデル・電子回路図・機械設計図を作成する.また,発砲スチロールを用いてプロトタイプの製作にあたる.その後,必要部品を調達・印刷し実際にロボットを組み立て,完成後、見直し作業・再設計・チューニングを行なっていく.
\begin{flushright}
(※文責:木島拓海)
\end{flushright}