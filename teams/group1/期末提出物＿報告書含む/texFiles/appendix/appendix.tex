% appendix.tex
% 付録
% ---Source---
\appendix
\chapter{}
\section{課題解決のための技術(新規取得)}
      \begin{itemize}
 \item  Fusion360をもちいた,3D CAD
 \item Arduinoによる各種センサの活用方法
      \end{itemize}
\section{課題解決のための技術(講義)}
      \begin{itemize}
 \item Arduinoによるモータの駆動(情報表現入門)
 \item KJ法(Communication I\hspace{-.1em}I\hspace{-.1em}I)
 \item ブレインストーミング(Communication I\hspace{-.1em}I\hspace{-.1em}I)
 \item リンク機構(ロボットの科学技術)
  \item  ロボット用センサ(ロボットの科学技術)

 \item ブレインストーミング
      \end{itemize}
\section{相互評価}
\begin{description}
   \item[伊藤壱]\mbox{}
      \begin{itemize}
       \item コメンター氏名:宮嶋佑\\
        \item コメンター氏名:藤内悠\\プロジェクトのリーダーを平行しつつグループの作業方針についても中心的な役割を果たし、方向性を指し示すことが多かったと思います。group1に限らずプロジェクト全体が計画性をもって作業できたのは伊藤君のおかげです。
 \item  コメンター氏名:宮嶋佑\\プロジェクトのリーダーを務めていながらも,グループ内でも率先してアイデアを出したり,意見を出していました.また,任された学習領域の電子回路の部分では,積極的に学習を進めて行ったり,知識の共有を行なっていました.\\
      \end{itemize}
         \item[木島拓海]\mbox{}
      \begin{itemize}
             \item コメンター氏名:伊藤壱\\
             木島君はどんな状況でも軽快に話をしてくれるので,多くの班員がその雰囲気に和まされたと思います.これからも持ち前の気前の良さでプロジェクトを支えてほしいと思います.
        \item コメンター氏名:藤内悠\\木島君は話し合いの場で方向性の確認や脱線をしないように適宜指摘をしてくれたと思います。また、活動の際に多角的な指摘で意見を出してくれた為、様々な間違いを早期に発見し非常に助かる場面が多くありました。
 \item  コメンター氏名:宮嶋佑\\グループ内での中間発表のスライド資料作りでは,的確な意見がもらえて助かりました.また,必要となった学習領域の割り当ての際,率先してそその学習領域に就いていました.\\
      \end{itemize}
         \item[藤内悠]\mbox{}
      \begin{itemize}
 \item  コメンター氏名:宮嶋佑\\ロボットの動きを考える時に,積極的に図示して説明していて,納得させられるところが多かったです.また,意見交換をする際に,率先して意見交換の場(docsなど)を開いてくれるので,円滑に物事を進めることができました.\\
  \item  コメンター氏名:伊藤壱\\
  藤内君は班員として励むだけではなく,プロジェクト全体の視点を持って熱心に取り組んでいました.その姿勢をとても尊敬しています.私がプロジェクトを進める上でとても助けられることが多かったと感謝しています.
      \end{itemize}
         \item[宮嶋佑]\mbox{}
      \begin{itemize}
      \item コメンター氏名:伊藤壱\\
   とても頑張っていたと思います.宮島さんの論理的な意見に何度も助けられました.責任感が強く最後まで仕事をやり抜く力を見習いたいと思います.
        \item コメンター氏名:宮嶋佑\\
        \item コメンター氏名:藤内悠\\話し合いや全体での作業が滞ってしまいそうな時に革新的なアイディアを提示し、参考になりそうな情報や資料を前もって準備する姿勢にはグループ全体として助けられたことが多くありました。
      \end{itemize}
\end{description}