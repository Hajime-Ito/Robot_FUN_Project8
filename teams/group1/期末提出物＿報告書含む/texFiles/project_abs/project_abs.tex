% project_abs.tex
% 第2章「プロジェクトの概要」の内容
% ---Source---
\chapter{プロジェクトの概要}
この章では,第1章を元に問題と課題の設定,それらの到達目標,到達目標を達成するための各個人の割り当てについて述べている.
\section{問題の設定}
1.1.2節で述べた既存のロボットの問題を以下のようにまとめた.
\begin{itemize}
    \item 外観が無機質で,威圧的に感じてしまう場合がある
    \item 客からの入力がない場合,静止する状態が続く.または,同じパターンの動きを繰り返してしまい,不気味に見えてしまう.
\end{itemize}
我々のグループは2つ目のロボットの「動き」に関わる問題に重点を置き活動をする.
\begin{flushright}
(※文責:宮嶋佑)
\end{flushright}
\section{課題の設定}
2.1節で述べた問題を,以下の制約条件下で解決することを考えた。
 \begin{itemize}
\item コロナウイルス感染対策を念頭に置き,対面での活動は必要最低限にする.
\item 低予算でかつ,高効率で安全に活動できる
\item 無意味な作業をなくす.
\item 大学の講義内で得た知識,技術を生かす.
\item 新たに学習を行い,大学の講義では得ることのできない知識,技術の習得も行う.
\end{itemize}
その結果,以下の具体策が提案された.
 \begin{itemize}
\item 各個人の作業を分担することで,作業の明確化を行う
\item   備品を購入する際には,金額や必要性をよく考え,自身の判断だけでなく,グループメンバーにも確認を取ってから,先生に備品購入の申請を行う.
\item 各個人で分担する作業には,今までの学習内容と,作業する上で新たに学習する必要がある領域の2つを含む.
\item KJ法を用いることで,多くの意見を引き出す.
\item ブレインストーミングを用いることで,効率的に関連性を見つける.
\end{itemize}
問題を解決するために,上記の具体例を活動課題とした.
\begin{flushright}
(※文責:宮嶋佑)
\end{flushright}
\section{到達レベル(目標)}
\subsection{ロボットの到達レベル(目標)}
グループ1では,店員の理想的な接客の「動き」を再現する.大きく2つの動作にわけ,さらにその中で2つの動き,合計で4つの動きの実装を目標として設定した.
\begin{description}
   \item[自発的な動作]\mbox{}
      \begin{itemize}
 \item  ロボットらしさを感じさせない,自然な動きの実装
 \item 押し付けがましくない,コンテンツの紹介を行う機能の実装\\
      \end{itemize}
   \item[客からの反応に答える動作]\mbox{}
      \begin{itemize}
      \item 客がいることを認識し,挨拶をする機能の実装
      \item ロボットが触られて,それに反応する機能の実装
         \end{itemize}
\end{description}
以上のロボット開発における目標を達成することで,客と自然なコミュニケーションを図ることのできるロボットの動きが実現できると考える.
\subsection{活動の到達レベル(目標)}
コロナウイルスの影響で,前年度とは全く違う活動方法となった.オンラインによる活動は新たな試みであり,わからない部分も多くある.また,対面の活動と比べ,コミュニケーションの取り方が非常に難しい.そのため,オンラインによる活動に目標を設定した.
\begin{itemize}
    \item 対面の活動よりも高頻度の報告と連絡をする.
    \item 音声だけではなく,画面共有やイラストを用い,コミュニケーションの相違をなくす.
    \item 活動を始める前にやるべきこと,終了する前に個人の進捗報告や意見交換の時間を設ける.
\end{itemize}
以上の活動における目標を常に意識することで,オンラインによる活動でも,円滑で間違いのない活動を行うことができると考える.
\begin{flushright}
(※文責:宮嶋佑)
\end{flushright}

\begin{comment}
メンバーに割り当てられた課題、課題を割り当てたプロセス
\end{comment}
\section{目標を達成するための割り当て}
意見交換を行い,以下の基準を提案しロボットを開発する上で,どの部分を担当するか,割り当てを行なった.
 \begin{itemize}
\item 各個人の得意分野
\item 各個人の興味のある分野
\item 各個人が習得したい技術
\item  作業負荷の均一性
\item パソコンのスペック(3D CAD を使用するには,ある程度のパソコンの動作環境が必要である.)
\end{itemize}
以上より,各個人の割り当ては以下のようになった.
\begin{description}
   \item[伊藤壱]\mbox{}
      \begin{itemize}
 \item  電子回路を中心に学習,設計
      \end{itemize}
   \item[木島拓海]\mbox{}
      \begin{itemize}
      \item リンク機構を中心とした,動きを実現する機構の学習,設計
         \end{itemize}
   \item[藤内悠]\mbox{}
      \begin{itemize}
      \item  歯車設計などを中心とした,動きを実現する機構の学習,設計
      \end{itemize}
   \item[宮嶋佑]\mbox{}   
   \begin{itemize}
   \item Fusion360による3D CADの学習,設計
   \end{itemize}
\end{description}
   また,割り当てごとに連携も行うことで,実現可能な動きか,実現するための変更点などの共有も行う.

\begin{flushright}
(※文責:宮嶋佑)
\end{flushright}
