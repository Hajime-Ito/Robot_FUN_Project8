% solution_processing.tex
% 第3章「課題解決のプロセス」の内容 藤内
% ---Source---
\chapter{課題解決のプロセス}
\begin{comment}
各々のメンバーに割り当てられた課題解決の方法とプロセスを記述する。
\end{comment}
\section{プロジェクト内における課題の位置付け}
\subsection{解決するべき課題}
今プロジェクトにおいて主に以下の二つの課題の解決を中心とした。
\begin{itemize}
\item 店員ロボットにおける理想の「動き」
\item ハードウェアによる実現
\end{itemize}
\begin{flushright}
(※文責:藤内悠)
\end{flushright}
\subsection{課題の持つ背景}
\noindent
 現在の店員ロボットは決してそのスペックの低さ故に取り扱わない店舗が多いというわけでは無い。むしろ有り余る性能を持つにも関わらず普及しているとは言い難い。その理由として従来の店員ロボットの動きに無機質さがあり、そのためお客さんには近寄り難い雰囲気を与えたり、それを設置する店側としてはかえって不利益を被るということがあるのでは無いかと考察した。
グループ1においては概要でも述べた通り「動作」に着目し、理想的かつ簡易で表現できる動きとは何か、またそれを再現する上でハードウェアに必要な要素として機構や外観の作りを考察するに至った。
そのためにまずは店員が果たすべき振る舞いとはどのようなものかを実際の店員の観察や認知心理に基づく理由を含め考察し、具体的にどのような動作があるのかといったことを実現可能な範囲で挙げること、その動作を全て実現可能とする店員ロボットの外観及び内面機構の設計が最終的な課題となった。
\begin{flushright}
(※文責:藤内悠)
\end{flushright}
\section{課題解決の方法}
\subsection{理想の「動き」への考察とプロセス}
 まず理想の店員を考察するにあたり、一言に店員と言えどそのあり方は多種多様である。例えばお客さんの質問を聞いてそれに応えるものもあれば、お客さんとの対話を通じて抽象的な要望を現実的な答えとして提示するものもある。具体的にどのような場における店員をモデルにするべきかを話し合い定義をした。
理想的な店員という抽象的な概念を各々の経験談を用いて情報を共有し、それらの店員がなぜ理想的と感じたかを分析することとした。
また、理想的な振る舞いに対してどのような動作が所謂「無機質」と感じられてしまい敬遠されてしまうかについての考察と議論を重ねた。
そこで一つの原因として待機状態において全く動作しないことであった。人間の店員であれば、お客さんとのコミュニケーションがない状態であっても何かしらの動作がある。それは何かしらの作業に取り組んでというだけではなく、お客さんからのコミュニケーションを待機するような状態でもある。
現実における理想的な店員の振る舞いではお客さんがそのような何か作業をしている店員であっても助言や意見を求めて店員が受動的にコミュニケーションを始めることが多い。しかしロボット店員ではそれがなされないことが多い。
しかしながら、店員ロボットが何かしらの作業を行っていた場合に話しかけるお客さんはあまりいない。さらに言えば某店員ロボットは「僕とお話ししようよ」と音声を流しつつ待機しているにも関わらず奇異の目で見られたり興味はあっても近寄られないということが多い。
そこで直接的にコミュニケーションを促すのではなく抱いている興味からその店を訪れたお客さんがそのロボットを見て思わず何をしているのかと気になって近づくような待機状態の動作が解決策になると考察した。加えて待機状態だけではなく当然コミュニケーションを図っている際にも無機質さを感じさせないような細かな所作として2.3.1で挙げた4つの動作を元として設定することとした。
\begin{flushright}
(※文責:藤内悠)
\end{flushright}
\subsection{ハードウェアによる実現への考察とプロセス}
自然かつ無機質でないような動きを第2章でも触れた自発的な動作と客からの反応に答える動作、それぞれに二種類の動作の計4種類の動作の具体的な動作をフローとして明確にした。
一つ一つの動作においてどのような条件が必要か、またその条件を取得するためのセンサ等はどの程度必要かの目星をある程度付けGoogleJamboardを用いて図示をおこなった。
その際に動きを再現するための機構や制御を複数の案を出しつつ選定・改善を行い各動作の一連の処理を決定するに至った。またそれと並行しつつ動作を無理なく再現できるようにロボットのハードウェアの側面で可能な工夫や内部の機構等を図面として起こし、身近な素材による簡易版や動きの再現を確認することで解決に取り組むこととなった。
\begin{flushright}
(※文責:藤内悠)
\end{flushright}