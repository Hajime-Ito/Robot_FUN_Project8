% introduction.tex
% 第1章「はじめに」の内容
% ---Source---
\chapter{はじめに} 
% 始めにとまとめを読んだだけで、全体が把握できるように記述する
%
この章では,現在のロボット型インターフェイスの現状とその問題,本プロジェクトで作成するロボット型インターフェイスの目的について記述する.
\begin{flushright}
(※文責:伊藤壱)
\end{flushright}
% 本文からはアラビア数字でページ番号を割り当てなおす。
\pagenumbering{arabic}
\section{ロボット型インタフェース}
\subsection{現状}
%
%
新型コロナウィルスの蔓延による社会情勢において,オンライン会議に代表されるような非接触のコミュニケーションが推進されている.小売業の実店舗において,人間同士の接客サービスが避けられる中,その代替手段としてロボットの接客に注目が集まっている.
これは,小売業の人手不足による店員ロボットの導入に拍車をかける形で需要を増やしている.
このような状況を受けて,ロボット産業の市場規模は2035年までに5倍となる見通しも出ている.
産業ロボットといえば,工場のオートメーション化に用いられるロボットが想起されるが,実際のところ、現在では店員ロボット導入の試みが至る所で行われている.
しかし,店員ロボットの分野はまだまだ未発達であり,ロボットの性能の問題や,顧客がロボットに馴染めないという意識的な問題を抱えている.
その理由として,店員ロボットは人間が働く環境で人間の行う仕事をそのまま引き受けて働くという状況にあり,工場で稼働されるような他の産業用ロボットよりも汎用的な性能を求められる点がある.
そのような店員ロボットの性能の問題を,人間が遠隔で操作をするという形で乗り越えるのが近年のトレンドになっているアバターロボットである.
アバターロボットとは,人間の身体の拡張であると捉えることが出来る.
しかし,現状でのアバターロボットは人間の指示の通りに動作をするのみであり,無機質さの解消には至っていない.
もし,人間に親しまれる機能をもつロボット型インタフェースを開発することが出来れば店員ロボットの導入の増加が期待されるほか,トレンドとなっているアバターロボットの実用性をさらに高めることが出来るだろう.
そこで私たちは,ロボットの持つ無機質さの解消に努め,顧客が馴染みやすい店員ロボットの開発を目指すことにした.
\begin{flushright}
(※文責:伊藤壱)
\end{flushright}
% プロジェクトの背景を記述する
\subsection{現状の問題}
%
%
産業界で工場などに導入されるロボットは生産技術を担う技能職の代替としての仕事が期待されるが,小売り業の実店舗などに導入される店員ロボットは接客を行うサービス職の代替としての仕事が期待される.
一般にロボットは無機質な外観をしており,人と同じかそれ以上大きく,素材も金属であったりする.
そのため,ロボットに対して威圧感を感じる人も少なくない.
これは,サービス業を担う店員ロボットとしては重大な問題である.
また,店員ロボット用に開発された機体においても外観の問題は解決されているが,人間からの入力がない場合に固まってしまったり,同じパターンの動作を無機質に繰り返してしまうという問題が見受けられる.
\begin{flushright}
(※文責:伊藤壱)
\end{flushright}
%\subsection{従来例}
\begin{comment}
現在の該当分野や類似プロジェクトの状況を記述する.
昨年のテーマを引き継いでいるプロジェクトでは、昨年の内容も記述する.
\end{comment}
%
%

%\begin{flushright}
%(※文責:伊藤壱)
%\end{flushright}
%\subsection{従来の問題}
%
%
\begin{comment}
目的を妨げている問題(現在の該当分野や類似プロジェクトの問題点)を記述する.
\end{comment}
%\begin{flushright}
%(※文責:伊藤壱)
%\end{flushright}
%\subsection{課題}
\begin{comment}
1.4節の従来の問題点の中から,解決すべき問題を明示し,その問題を解決するために設定された課題(解決すべき問題を具体的に記述したもの)の概略を記述する.
\end{comment}
%
%
%\begin{flushright}
%(※文責:伊藤壱)
%\end{flushright}
\section{今回開発したロボット型インタフェース}
まだ開発していないため,記述できません.
\begin{flushright}
(※文責:伊藤壱)
\end{flushright}
\section{目的}
\begin{comment}
プロジェクトの目的を記述する.ここでの目的とは,最も高位のものであり,意思や希望を表す.
プロジェクトのテーマの説明にもなるように記述する.
\end{comment}
%
%
1.1.1項で述べた通り,小売業などでの店員ロボットの導入は進んでおり,アバターロボットのような形態での導入がトレンドになりつつある.しかし、1.1.2項で述べたようにロボットの持つ無機質さの解消が大きい課題となっている.そこで本プロジェクトでは,有用性のある店員ロボットを目指しつつ,人間が馴染みやすいロボットの動き,機能の充実を実現する.
\begin{flushright}
(※文責:伊藤壱)
\end{flushright}